\documentclass[10pt]{beamer}

%%% effets speciaux
 %Profondeur&de&champ&(Depth&Of&Field)&
% God&rays&(light&scaFering)&
% Glows&(blooming)&

\usetheme[progressbar=frametitle]{m}

\usepackage{booktabs}
\usepackage[scale=2]{ccicons}
\usepackage{hyperref}

\usepackage{pgfplots}
\usepgfplotslibrary{dateplot}

\title{Les effets spéciaux dans les jeux vidéo}
\subtitle{Moteur de jeu}
\date{27 novembre 2015}
\author{Thibaut Castanié, Vincent Bazia}
\institute{Master IMAGINA - Université Montpellier}

\begin{document}

\maketitle

\begin{frame}
  \frametitle{Table of Contents}
  \setbeamertemplate{section in toc}[sections numbered]
  \tableofcontents[hideallsubsections]
\end{frame}

\section{Introduction}

%% http://graphics.berkeley.edu/papers/Kosloff-AAF-2007-08/Kosloff-AAF-2007-08.pdf

% http://www.gamasutra.com/view/feature/3102/four_tricks_for_fast_blurring_in_.php?print=1

%% http://www.visus.uni-stuttgart.de/uploads/tx_vispublications/eg07-kraus.pdf
\section{Depth of field}
\begin{frame}
	\frametitle{Depth of field : Présentation}
    \begin{center}
	\includegraphics[scale=0.6]{images/starwars.jpg}
	\end{center}
\end{frame}

\begin{frame}
	\frametitle{Depth of field : Mise en oeuvre}
    \begin{center}
    \includegraphics[scale=0.5]{images/noblur.jpg}\\
    \pause
    \includegraphics[scale=0.5]{images/zbuf.png}
	\end{center}
    
\end{frame}

\begin{frame}
	\frametitle{Depth of field : Mipmaps}
    \begin{center}
    \includegraphics[scale=0.6]{images/mipmaps.png}
	\end{center}
\end{frame}

\begin{frame}
	\frametitle{Depth of field : Résultat}
    \begin{center}
    \includegraphics[scale=0.42]{images/blur.jpg}
    \hspace{0.2cm}
    \pause
    \includegraphics[scale=0.42]{images/noblur.jpg}
	\end{center}
\end{frame}


\section{Glows (blooming)}

\begin{frame}
	\frametitle{Glows : Présentation}
    \centering
    \includegraphics[scale=0.45]{images/bloom.png}
\end{frame}

\begin{frame}
	\frametitle{Glows : Présentation}
    \centering
    \includegraphics[scale=0.35]{images/bloom.jpg}
\end{frame}

% http://www.gamasutra.com/view/feature/130520/realtime_glow.php?page=1
\begin{frame}
	\frametitle{Glows : Fonctionnement}
    \centering
    \includegraphics[scale=0.7]{images/bloomTex.jpeg}\\
    \vspace{0.5cm}\pause
    \includegraphics[scale=0.7]{images/bloomplus.jpg}
\end{frame}




\section{God rays}
% http://research.ijcaonline.org/volume108/number11/pxc3900275.pdf

% http://www.gamasutra.com/blogs/BartlomiejWronski/20141208/226295/Atmospheric_scattering_and_volumetric_fog_algorithm__part_1.php
\begin{frame}
	\frametitle{God rays : Présentation}
    \centering
    \includegraphics[scale=0.45]{images/gr1.jpg}\\
    Exemple : \url{http://threejs.org/examples/webgl_postprocessing_godrays.html}
\end{frame}

\begin{frame}
	\frametitle{God rays : Mise en oeuvre}
    \centering
    \includegraphics[scale=0.65]{images/gr1.png}
\end{frame}

\begin{frame}
	\frametitle{God rays : Mise en oeuvre}
    \centering
    \includegraphics[scale=0.5]{images/gr2.png}
\end{frame}

\begin{frame}
	\frametitle{God rays : Mise en oeuvre}
    \centering
    \includegraphics[scale=0.5]{images/gr3.png}
\end{frame}

\begin{frame}
	\frametitle{God rays : Mise en oeuvre}
    \centering
    \includegraphics[scale=0.5]{images/gr4.png}
\end{frame}

\begin{frame}
	\frametitle{God rays : Résultat}
    \centering
    \includegraphics[scale=0.5]{images/gr5.png}
\end{frame}

\begin{frame}
	\frametitle{God rays : Comparatif}
    \centering
    \includegraphics[scale=0.3]{images/gr5.png}
    \vspace{0.5cm}
    \includegraphics[scale=0.4]{images/gr1.png}
\end{frame}

\section{Motion blur}

\begin{frame}
	\frametitle{Motion blur : Présentation}
    \centering
    \includegraphics[scale=0.5]{images/motionblur2.jpg}
\end{frame}


\begin{frame}
	\frametitle{Motion blur : Son utilisation dans le jeu vidéo}
    \centering
    \includegraphics[scale=0.45]{images/motionblur3.jpg}
\end{frame}

\begin{frame}
	\frametitle{Motion blur : Son utilisation dans le jeu vidéo}
    \centering
    \includegraphics[scale=0.2]{images/motionblur.jpg}
\end{frame}

\begin{frame}
	\frametitle{Motion blur : Fonctionnement}
    Deux méthodes principales\pause
    \begin{center}
	\includegraphics[scale=0.28]{./images/blur1.png}\\\pause
	\includegraphics[scale=0.3]{./images/blur2.png}
	\end{center}
\end{frame}

\section{Conclusion}

\begin{frame}
	\frametitle{Bibliographie}
    
    \begin{thebibliography}{}
    
    \bibitem{} 
    \textit{n}Vidia, Randima Fernando, \textit{GPU Gems 1}, Addison-Wesley Professional, 2004

    \bibitem{} 
    \textit{n}Vidia, Hubert Nguyen, \textit{GPU Gems 3}, Addison-Wesley Professional, 2007
    
    \bibitem{}
    Julien Moreau-Mathis, \textit{Gods Rays? What's that?},  \url{medium.com}, 2014

    \bibitem{}
    Unreal Engine 4 documention,
    \\\url{https://docs.unrealengine.com/}
    
    \end{thebibliography}


\end{frame}

\plain{Des questions ?}

\end{document}
