\documentclass[a4paper,11pt]{article}

\usepackage[french]{babel}
\usepackage[T1]{fontenc}
\usepackage[utf8]{inputenc}
\usepackage{graphicx}
\usepackage{hyperref}
%\usepackage{fullpage}

\begin{document}

\title{\textbf{Compte rendu du TP \no 1}\\Méhode de binarisation}
\author{Thibaut Castanié\\\textit{M2 IMAGINA}}
\date{3 octobre 2015}

\maketitle
\thispagestyle{empty}

\newpage 

\section{QR Code de base}

\vspace{1cm}

\begin{center}
\includegraphics[scale=0.5]{./v2.png}\\
\textit{L'image originale utilisée pour le TP}
\end{center}

\section{Binarisation simple}

Afin de binariser l'image, on utilise le seuil suivant : $Th=(p_{max}-p_{min})/2$
On obtient le résultat suivant :

\begin{center}
\includegraphics[scale=0.5]{./v2_4x4.png} \hspace{3cm}
\includegraphics[scale=0.5]{./v2_4x4binarise.png}\\
\textit{QR code original} \hspace{2cm} \textit{QR code binarisé}
\end{center}

\section{Binarisation par vote majoritaire}

Pour obtenir un QR code plus "propre" et plus facile à analyser, nous appliquons la même fonction que précédemment, mais cette fois à chaque bloc de 4x4 pixels, en appliquant un vote majoritaire à chacun.

\begin{center}
\includegraphics[scale=0.5]{./v2_4x4.png} \hspace{3cm}
\includegraphics[scale=0.5]{./v2binarise.png}\\
\textit{QR code original} \hspace{1.3cm} \textit{QR code binarisé par vote}
\end{center}

\section{Binarisation locale}

On utilise l'algorithme de Niblack où $Th = \mu *(1 - k *(1-\sigma/R))$

\begin{center}
\includegraphics[scale=0.5]{./v2_4x4.png} \hspace{3cm}
\includegraphics[scale=0.5]{./v2binarise.png}\\
 \hspace{0.5cm} \textit{QR code original} \hspace{1cm} \textit{QR code binarisé localement}
\end{center}

\end{document}